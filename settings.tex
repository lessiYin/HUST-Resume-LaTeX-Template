\documentclass[11pt]{article}
\usepackage{xltxtra}
\usepackage{bookmark}
\usepackage{hyperref}
\hypersetup{hidelinks}
\usepackage{url}
\urlstyle{tt}
\usepackage{multicol}
\usepackage{xcolor}
\usepackage{calc}
\usepackage{graphicx}
\usepackage{tikz}
\usetikzlibrary{calc}
\usepackage{fontspec}
\usepackage{xeCJK}
\usepackage{relsize}
\usepackage{xspace}
\usepackage{fontawesome5}
\usepackage{titlesec}
\usepackage{enumitem}
\usepackage{siunitx}
\usepackage{amssymb}
\usepackage{tabularx}
\usepackage{multicol}
\usepackage{fontspec}
\usepackage{fancybox}
\usepackage{float}

% 一些小设置
\CJKsetecglue{}
\protected\def\Cpp{{C\nolinebreak[4]\hspace{-.05em}\raisebox{.28ex}{\relsize{-1}++}}\xspace}
\setlength{\parindent}{0pt}
\pagenumbering{gobble}
\setlist[itemize]{topsep=0em, leftmargin=*}
\setlist[enumerate]{topsep=0em, leftmargin=*}

% --- 华中科技大学配色 ---
% 官方标准红 RGB: 193, 0, 22 (大致)
% 为了简历阅读舒适,这里稍微调暗一点点,显得稳重
\definecolor{HUST_Red}{RGB}{176, 31, 36} 
% 如果你喜欢原来那种蓝色风格,也可以保留蓝色,华科理工科用深蓝也很合适
% \definecolor{HUST_Blue}{RGB}{0, 51, 102} 

% --- 标题格式设置 (使用 HUST_Red) ---
\titleformat{\section}
  {\LARGE\bfseries\raggedright}
  {}{0em}
  {}
  [{\color{HUST_Red}\titlerule}] % 标题下划线颜色
\titlespacing*{\section}{0cm}{*1.2}{*1.2}

\titleformat{\subsection}
  {\large\bfseries\raggedright}
  {}{0em}
  {}
  []
\titlespacing*{\subsection}{0cm}{*1.2}{*1.2}

% 页面大小与页边距
\usepackage[
	a4paper,
	left=1.2cm,
	right=1.2cm,
	top=1.8cm, % 稍微增加顶部距离以容纳页眉
	bottom=1cm,
	nohead
]{geometry}

\renewcommand{\arraystretch}{1.2}
\linespread{1.2}
\renewcommand{\CJKglue}{\hskip 0.05em}

% --- 字体设置 (优化版) ---
% 原模板使用了特定路径的字体,如果在 Overleaf 或没有该字体会报错。
% 这里做了判断:如果能找到 SweiSpring 就用,找不到就用系统自带的。

\ifxetex
  % 尝试加载自定义字体,如果不想用自定义字体,直接注释掉下面这块
  % \setmainfont[Path=fonts/,Extension=.ttf,BoldFont=* Bold]{SweiSpring}
  % \setCJKmainfont[Path=fonts/,Extension=.ttf,BoldFont=* Bold]{SweiSpring}
  
  % --- 推荐方案:使用通用字体 (Overleaf 上可用 Fandol 系列) ---
  % 如果你在本地 Windows,可以改为 SimSun (宋体) 和 SimHei (黑体)
  \setCJKmainfont[BoldFont=FandolHei-Regular, ItalicFont=FandolKai-Regular]{FandolSong-Regular}
  \setmainfont{Times New Roman} % 英文用 Times 比较正式
\fi